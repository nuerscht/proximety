In diesem Kapitel werden die Testing Ideen gesammelt und erläutert sowie die zu verwendenden Resourcen definiert.

\subsection{Test Ideen und Vorgehen}
Unit-Tests können für kritische oder komplexe Komponenten verwendet werden sind aber keine Pflicht (in Absprache mit dem Dozenten). 
Im Allgemeinen wollen wir uns auf die System und UI Tests fokussieren um den Aufwand des Projekts im Rahmen zu halten.

\medskip
Manuelle Tests sind zwar üblicherweise einfach zu erstellen, müssen oder sollten aber bei jeder Änderung des Projekts wiederholt werden. Dies führt schnell zu viel Aufwand. 
Automatisierte Tests hingegen sind in der Konzeption oft aufwändiger, sparen dafür aber bei der Ausführung Zeit, solange sie nicht angepasst werden müssen.

\medskip
Als Richtlinie sollen alle definierten Use Cases mit mindestens einem Test Case abgedeckt werden. 
Um die ortsabhängigen Fälle zu prüfen soll die Android Mock Location Funktionalität verwendet werden (siehe Resourcen Abschnitt~\ref{sec:test_resources})

\subsection{Test Resourcen}
\label{sec:test_resources}
Android UI Automation: \url{http://developer.android.com/tools/testing/testing_ui.html}

Mock Locations in Android: \url{http://developer.android.com/training/location/location-testing.html}

Um die Position auf dem zu testenden Gerät zu setzten wird eine zusätzliche App benötigt. Es bietet sich an, eine bereits verfügbar aus dem Google Play Store zu verwenden:
Fake GPS Location Spoofer Free: \url{https://play.google.com/store/apps/details?id=com.incorporateapps.fakegps.fre}

\medskip

Als Test-Geräte wird der Android Emulator (Android 4.4/5.0) sowie die vorhandenen Geräte der Projektmitglieder verwendet. 

\medskip
Folgende Umgebungen sind verfügbar. Eine Abkürzung für spätere referenzierung wird jeweils gegeben:

\begin{itemize}
\item Android Emulator (Android 4.4) [AE-44]
\item Samsung Galaxy Nexus (Android 4.3) [GNEX-43]
\item Android Emulator (Android 5.0) [AE-50]
\item Google/LG Nexus 5 (Android 5.0) [N5-50]
\end{itemize}

\newpage
\subsection{Manuell Test Fälle}

\begin{center}
	\begin{tabular}{ | l | l | l | l | }
	\hline
	  \textbf{ID} & \textbf{Name} & \textbf{Priorität} & \textbf{Abdeckung} \\
	  \hline
	  \hypertarget{TC01}{TC01} & Login / Registrieren & Hoch & UC01 \\
	  \hline
	  \mc{\textbf{Vorbedingung}} & \mc{\textbf{Nachbedingung}} \\
	  \hline
	  \mc{Abgemeldet} & \mc{Account erstellt und angemeldet} \\
	  \hline
	  \mc{\textbf{Schritte}} & \mc{\textbf{Erwartet}} \\
	  \hline
	  \mc{ App starten } & \mc{} \\
	  \mc{ Registrieren drücken } & \mc{ Registrier Formular erscheint } \\
	  \mc{ Namen, E-Mail und Passwort eingeben (2x) } & \mc{} \\
	  \mc{ Registrieren drücken } & \mc{ Login Formular erscheint } \\
	  \mc{ Vorher definiertes Passwort eingeben } & \mc{ Login erfolgreich, Popup erscheint um Freunde hinzuzufügen } \\
	  \hline
	\end{tabular}
\end{center}

\begin{center}
	\begin{tabular}{ | l | l | l | l | }
	\hline
	  \textbf{ID} & \textbf{Name} & \textbf{Priorität} & \textbf{Abdeckung} \\
	  \hline
	  \hypertarget{TC02}{TC02} & Freundschaftsanfrage (Ablehnen) & Mittel & UC02 \\
	  \hline
	  \mc{\textbf{Vorbedingung}} & \mc{\textbf{Nachbedingung}} \\
	  \hline
	  \mc{Abgemeldet, 2. Account bereit (Freund)} & \mc{Anfrage versendet und bestätigt} \\
	  \hline
	  \mc{\textbf{Schritte}} & \mc{\textbf{Erwartet}} \\
	  \hline
	  \mc{ Drawer Menu öffnen -> ``Freund hinzufügen`` auswählen } & \mc{Formular erscheint} \\
	  \mc{ E-Mail des Freundes eingeben, Einladen auswählen } & \mc{Einladung versendet, zurück auf Hauptbildschirm} \\
	  \mc{ Drawer Menu öffnen -> ``Abmelden`` auswählen } & \mc{ Register/Login erscheint } \\
	  \mc{ Mit ``Freund`` anmelden } & \mc{ Main Screen } \\
	  \mc{ Drawer Menu öffnen -> ``Offene Anfragen`` auswählen } & \mc{ Liste mit Anfragen erscheint } \\
	  \mc{ Anfrage ablehnen } & \mc{ Zurück auf Hauptbildschirm, Freund wird nicht angezeigt } \\
	  \hline
	\end{tabular}
\end{center}







\begin{center}
	\begin{tabular}{ | l | l | l | l | }
	\hline
	  \textbf{ID} & \textbf{Name} & \textbf{Priorität} & \textbf{Abdeckung} \\
	  \hline
	  \hypertarget{TC04}{TC04} & Freundschaftsanfrage (Annehmen) & Hoch & UC02 \\
	  \hline
	  \mc{\textbf{Vorbedingung}} & \mc{\textbf{Nachbedingung}} \\
	  \hline
	  \mc{Angemeldet, 2. Account bereit (Freund)} & \mc{Anfrage versendet und bestätigt} \\
	  \hline
	  \mc{\textbf{Schritte}} & \mc{\textbf{Erwartet}} \\
	  \hline
	  \mc{ Drawer Menu öffnen -> ``Freund hinzufügen`` auswählen } & \mc{Formular erscheint} \\
	  \mc{ E-Mail des Freundes eingeben, Einladen auswählen } & \mc{Einladung versendet, zurück auf Hauptbildschirm} \\
	  \mc{ Drawer Menu öffnen -> ``Abmelden`` auswählen } & \mc{ Register/Login erscheint } \\
	  \mc{ Mit ``Freund`` einloggen } & \mc{ Main Screen } \\
	  \mc{ Drawer Menu öffnen -> ``Offene Anfragen`` auswählen } & \mc{ Liste mit Anfragen erscheint } \\
	  \mc{ Anfrage annehmen } & \mc{ Zurück auf Hauptbildschirm, Freund wird angezeigt } \\
	  \hline
	\end{tabular}
\end{center}

\begin{center}
	\begin{tabular}{ | l | l | l | l | }
	\hline
	  \textbf{ID} & \textbf{Name} & \textbf{Priorität} & \textbf{Abdeckung} \\
	  \hline
	  \hypertarget{TC05}{TC05} & Freundschaft Auflösen & Hoch & UC03 \\
	  \hline
	  \mc{\textbf{Vorbedingung}} & \mc{\textbf{Nachbedingung}} \\
	  \hline
	  \mc{Angemeldet, mind 1 Freund} & \mc{Freundschaft aufgelöst} \\
	  \hline
	  \mc{\textbf{Schritte}} & \mc{\textbf{Erwartet}} \\
	  \hline
	  \mc{ Freund aus Liste auswählen } & \mc{Freund Details werden angezeigt} \\
	  \mc{ Reiter Details ``Freund entfernen`` auswählen } & \mc{Benachrichtigung, zurück auf Hauptbildschirm, Freund erscheint nicht mehr in der Liste} \\
	  \hline
	\end{tabular}
\end{center}

\begin{center}
	\begin{tabular}{ | l | l | l | l | }
	\hline
	  \textbf{ID} & \textbf{Name} & \textbf{Priorität} & \textbf{Abdeckung} \\
	  \hline
	  \hypertarget{TC06}{TC06}  & Aktuelle Position Abfragen & Hoch & UC04 \\
	  \hline
	  \mc{\textbf{Vorbedingung}} & \mc{\textbf{Nachbedingung}} \\
	  \hline
	  \mc{Angemeldet, mind 1 Freund} & \mc{Position des Freundes bekannt} \\
	  \hline
	  \mc{\textbf{Schritte}} & \mc{\textbf{Erwartet}} \\
	  \hline
	  \mc{ Freund aus Liste auswählen } & \mc{Freund Details werden angezeigt} \\
	  \mc{ Reiter Details Distanz } & \mc{Aktuelle geschätze Distanz wird angezeigt} \\
	  \mc{ Reiter Karte auswählen } & \mc{Aktuelle Position wird auf der Karte angezeigt} \\
	  \hline
	\end{tabular}
\end{center}

\begin{center}
	\begin{tabular}{ | l | l | l | l | }
	\hline
	  \textbf{ID} & \textbf{Name} & \textbf{Priorität} & \textbf{Abdeckung} \\
	  \hline
	  \hypertarget{TC07}{TC07} & Proximety Alarm & Hoch & UC05, UC08 \\
	  \hline
	  \mc{\textbf{Vorbedingung}} & \mc{\textbf{Nachbedingung}} \\
	  \hline
	  \mc{Angemeldet, mind 1 Freund, Freund hat bekannte Position, Mock Location App installiert (siehe Resourcen sowie Anweisungen der App)} & \mc{Proximety Alarm ausgelöst} \\
	  \hline
	  \mc{\textbf{Schritte}} & \mc{\textbf{Erwartet}} \\
	  \hline
	  \mc{ Mock Location mit zusätzlicher App auf entfernten Ort setzten (Bsp. Sydney) } & \mc{Kein Alarm} \\
	  \mc{ Proximety Drawer Menu öffnen -> ``Einstellungen`` auswählen } & \mc{} \\
	  \mc{ Distanz für Alarm 10 km einstellen } & \mc{} \\
	  \mc{ Mock Location auf Ort in der nähe des Freundes setzten (~ 8km) } & \mc{(Warten) Proximety Alarm Notification wird angezeigt} \\
	  \mc{ Klick auf die Notification } & \mc{Detail Ansicht des Freundes wird geöffnet} \\
	  \mc{ Mock Location mit zusätzlicher App auf entfernten Ort setzten (Bsp. Hamburg) } & \mc{Kein Alarm} \\
	  \mc{ Proximety Drawer Menu öffnen -> ``Einstellungen`` auswählen } & \mc{} \\
	  \mc{ Distanz für Alarm reduzieren auf 3 km } & \mc{} \\
	  \mc{ Mock Location auf Ort in der nähe des Freundes setzten < 2km } & \mc{(Warten) Proximety Alarm Notification wird angezeigt} \\
	  \hline
	\end{tabular}
\end{center}

\begin{center}
	\begin{tabular}{ | l | l | l | l | }
	\hline
	  \textbf{ID} & \textbf{Name} & \textbf{Priorität} & \textbf{Abdeckung} \\
	  \hline
	  \hypertarget{TC08}{TC08} & Freund kontaktieren & Tief & UC06 \\
	  \hline
	  \mc{\textbf{Vorbedingung}} & \mc{\textbf{Nachbedingung}} \\
	  \hline
	  \mc{Angemeldet, mind 1 Freund} & \mc{E-Mail an Freund gesendet} \\
	  \hline
	  \mc{\textbf{Schritte}} & \mc{\textbf{Erwartet}} \\
	  \hline
	  \mc{ Freund aus Liste auswählen } & \mc{Freund Details werden angezeigt} \\
	  \mc{ Kontakt Icon auswählen } & \mc{Externe Applikation für E-Mail wird gestartet (Bsp. GMail), E-Mail von Freund eingetragen} \\
	  \mc{ Nachricht schreiben und absenden } & \mc{Freund erhält geschrieben Nachricht} \\
	  \hline
	\end{tabular}
\end{center}

\begin{center}
	\begin{tabular}{ | l | l | l | l | }
	\hline
	  \textbf{ID} & \textbf{Name} & \textbf{Priorität} & \textbf{Abdeckung} \\
	  \hline
	  \hypertarget{TC09}{TC09} & Freund-Spezifische Einstellungen & Mittel & UC07 \\
	  \hline
	  \mc{\textbf{Vorbedingung}} & \mc{\textbf{Nachbedingung}} \\
	  \hline
	  \mc{Angemeldet, mind 1 Freund} & \mc{Einstellungen angepasst} \\
	  \hline
	  \mc{\textbf{Schritte}} & \mc{\textbf{Erwartet}} \\
	  \hline
	  \mc{ Freund aus Liste auswählen } & \mc{Freund Details werden angezeigt} \\
	  \mc{ Alarm abschalten } & \mc{} \\
	  \mc{ TC07 ausführen } & \mc{Es wird kein Alarm ausgelöst} \\
	  \hline
	\end{tabular}
\end{center}

\begin{center}
	\begin{tabular}{ | l | l | l | l | }
	\hline
	  \textbf{ID} & \textbf{Name} & \textbf{Priorität} & \textbf{Abdeckung} \\
	  \hline
	  \hypertarget{TC10}{TC10} & Allgemeine Einstellungen & Mittel & UC08 \\
	  \hline
	  \mc{\textbf{Vorbedingung}} & \mc{\textbf{Nachbedingung}} \\
	  \hline
	  \mc{Angemeldet} & \mc{Einstellungen angepasst} \\
	  \hline
	  \mc{\textbf{Schritte}} & \mc{\textbf{Erwartet}} \\
	  \hline
	  \mc{ Drawer Menu öffnen -> ``Einstellungen`` auswählen } & \mc{Einstellungen erscheinen} \\
	  \mc{ Distanz auf 11 km ändern } & \mc{} \\
	  \mc{ Allg-Alarm ausschalten } & \mc{} \\
	  \mc{ Abmelden, App schliessen } & \mc{} \\
	  \mc{ App starten, Anmelden } & \mc{Hauptbildschirm erscheint} \\
	  \mc{ Drawer Menu öffnen -> ``Einstellungen`` auswählen } & \mc{Einstellungen gespeichert, wie in Schritt 2 \& 3} \\
	  \hline
	\end{tabular}
\end{center}

\newpage
\subsection{Test Resulate}
Hier werden die Resulatate der Ausführungen mit der verwendeten Umgebung aufgeführt werden.

\begin{center}
	\begin{tabular}{ | l | l | l | l | p{7cm} | }
	\hline
	Name / Id & Datum & Umgebung & Resultat & Bemerkungen \\
	\hline
	\hyperlink{TC01}{TC01} & 14.1.2015 & N5-50 & \textcolor{green}{OK} & \\
	\hline
	\hyperlink{TC02}{TC02} & 14.1.2015 & N5-50 & \textcolor{green}{OK} & \\
	\hline
	\hyperlink{TC03}{TC03} & 14.1.2015 & N5-50 & \textcolor{green}{OK} & \\
	\hline
	\hyperlink{TC04}{TC04} & 14.1.2015 & N5-50 & \textcolor{green}{OK} & \\
	\hline
	\hyperlink{TC05}{TC05} & 14.1.2015 & N5-50 & \textcolor{green}{OK} & \\
	\hline
	\hyperlink{TC06}{TC06} & 14.1.2015 & N5-50 & \textcolor{green}{OK} & \\
	\hline
	\hyperlink{TC07}{TC07} & 14.1.2015 & N5-50 & \textcolor{red}{NOK} & Einstellung der Distanz nicht möglich. Fix 5km. Alarm funktioniert jedoch mit Notification \\
	\hline
	\hyperlink{TC08}{TC08} & 14.1.2015 & N5-50 & \textcolor{green}{OK} & \\
	\hline
	\hyperlink{TC09}{TC09} & 14.1.2015 & N5-50 & \textcolor{red}{NOK} & Alarm erscheint trotzdem. Einstellung wird nicht gespeichert\\
	\hline
	\hyperlink{TC10}{TC10} & 14.1.2015 & N5-50 & \textcolor{red}{NOK} & Einstellungen werden gespeichert. Es gibt jedoch keine Distanz Einstellung\\
	\hline
	\hyperlink{TC01}{TC01} & 15.1.2015 & GNEX-43 & \textcolor{green}{OK} & \\
	\hline
	\hyperlink{TC02}{TC02} & 15.1.2015 & GNEX-43 & \textcolor{green}{OK} & \\
	\hline
	\hyperlink{TC03}{TC03} & 15.1.2015 & GNEX-43 & \textcolor{green}{OK} & \\
	\hline
	\hyperlink{TC04}{TC04} & 15.1.2015 & GNEX-43 & \textcolor{green}{OK} & \\
	\hline
	\hyperlink{TC05}{TC05} & 15.1.2015 & GNEX-43 & \textcolor{green}{OK} & \\
	\hline
	\hyperlink{TC06}{TC06} & 15.1.2015 & GNEX-43 & \textcolor{green}{OK} & \\
	\hline
	\hyperlink{TC07}{TC07} & 15.1.2015 & GNEX-43 & \textcolor{green}{OK} & Einstellung der Distanz pro Freund möglich \\
	\hline
	\hyperlink{TC08}{TC08} & 15.1.2015 & GNEX-43 & \textcolor{green}{OK} & \\
	\hline
	\hyperlink{TC09}{TC09} & 15.1.2015 & GNEX-43 & \textcolor{red}{NOK} & Alarm erscheint trotzdem. Einstellung wird nicht gespeichert\\
	\hline
	\hyperlink{TC10}{TC10} & 15.1.2015 & GNEX-43 & \textcolor{green}{OK} & Einstellung der Distanz pro Freund möglich\\
	\hline
	\hyperlink{TC07}{TC07} & 16.1.2015 & N5-50 & \textcolor{green}{OK} & \\
	\hline
	\hyperlink{TC09}{TC09} & 16.1.2015 & N5-50 & \textcolor{green}{OK} & \\
	\hline
	\hyperlink{TC10}{TC10} & 16.1.2015 & N5-50 & \textcolor{green}{OK} & Einstellung der Distanz pro Freund möglich\\
	\hline
	\hyperlink{TC09}{TC09} & 16.1.2015 & GNEX-43 & \textcolor{green}{OK} & \\
	\hline
	\end{tabular}
\end{center}



