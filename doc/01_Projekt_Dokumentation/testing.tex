In diesem Kapitel werden die Testing Ideen gesammelt und erläutert sowie die zu verwendenden Resourcen definiert.

\subsection{Test Ideen und Vorgehen}
Unit-Tests können für kritische oder komplexe Komponenten verwendet werden sind aber keine Pflicht (in Absprache mit dem Dozenten). 
Im Allgemeinen wollen wir uns auf die System und UI Tests fokussieren um den Aufwand des Projekts im Rahmen zu halten.

\medskip
Manuelle Tests sind zwar üblicherweise einfach zu erstellen, müssen oder sollten aber bei jeder Änderung des Projekts wiederholt werden. Dies führt schnell zu viel Aufwand, darum sollen alle Tests so weit als möglich automatisiert werden.

\medskip
Als Richtlinie sollen alle definierten Use Cases mit mindestens einem Test Case abgedeckt werden. Um die ortsabhängigen Fälle zu prüfen soll die Android Mock Location Funktionalität verwendet werden (siehe Resourcen Abschnitt~\ref{sec:test_resources})
Des weitern soll die Validierung bei allen Eingabemöglichkeiten mit positiven sowie negativen Tests geprüft werden.

\medskip
Die Kommunikation zum Server wird mit Service Mocks auf der Server Seite geprüft.

\subsection{Test Resourcen}
\label{sec:test_resources}
Android UI Automation: \url{http://developer.android.com/tools/testing/testing_ui.html}

Fake/Mock Locations: \url{http://developer.android.com/training/location/location-testing.html}

\medskip

Als Test-Geräte wird der Android Emulator (Android 4.4) sowie die vorhandenen Geräte der Projektmitglieder verwendet. 

\medskip
Folgende Umgebungen sind verfügbar. Eine Abkürzung für spätere referenzierung wird jeweils gegeben:

\begin{itemize}
\item Android Emulator (Android 4.4) [AE-44]
\item Google/LG Nexus 5 (Android 5.0) [N5-50]
\item Samsung Galaxy Note 4 (Android 4.4) [Note-44]
\item Samsung Galaxy Nexus (Android 4.3) [GN-43]
\end{itemize}

\subsection{Test Fälle}
Referenzen zu automatisierten Test Fällen
\begin{center}
	\begin{tabular}{ | l | l | l | l | }
	\hline
	Name / Id & Priorität & Referenz & Abdeckung \\
	\hline
	 & & & \\
	\hline
	\end{tabular}
\end{center}

Manuelle Test Fälle:

\begin{center}
	\begin{tabular}{ | l | l | l | l | }
	\hline
	  \textbf{ID} & \textbf{Name} & \textbf{Priorität} & \textbf{Abdeckung} \\
	  \hline
	  TC01 & Login / Registrieren & Hoch & UC01 \\
	  \hline
	  \mc{\textbf{Vorbedingung}} & \mc{\textbf{Nachbedingung}} \\
	  \hline
	  \mc{Abgemeldet} & \mc{Account erstellt und angemeldet} \\
	  \hline
	  \mc{\textbf{Schritte}} & \mc{\textbf{Erwartet}} \\
	  \hline
	  \mc{ App starten } & \mc{} \\
	  \mc{ Registrieren drücken } & \mc{ Registrier Formular erscheint } \\
	  \mc{ Namen, E-Mail und Passwort eingeben (2x) } & \mc{} \\
	  \mc{ Registrieren drücken } & \mc{ Login Formular erscheint } \\
	  \mc{ Vorher definiertes Passwort eingeben } & \mc{ Login erfolgreich, Popup erscheint um Freunde hinzuzufügen } \\
	  \hline
	\end{tabular}
\end{center}

\begin{center}
	\begin{tabular}{ | l | l | l | l | }
	\hline
	  \textbf{ID} & \textbf{Name} & \textbf{Priorität} & \textbf{Abdeckung} \\
	  \hline
	  TC02 & Freundschaftsanfrage (Ablehnen) & Mittel & UC02 \\
	  \hline
	  \mc{\textbf{Vorbedingung}} & \mc{\textbf{Nachbedingung}} \\
	  \hline
	  \mc{Abgemeldet, 2. Account bereit (Freund)} & \mc{Anfrage versendet und bestätigt} \\
	  \hline
	  \mc{\textbf{Schritte}} & \mc{\textbf{Erwartet}} \\
	  \hline
	  \mc{ Drawer Menu öffnen -> ``Freund hinzufügen`` auswählen } & \mc{Formular erscheint} \\
	  \mc{ E-Mail des Freundes eingeben, Einladen auswählen } & \mc{Einladung versendet, zurück auf Hauptbildschirm} \\
	  \mc{ Drawer Menu öffnen -> ``Abmelden`` auswählen } & \mc{ Register/Login erscheint } \\
	  \mc{ Mit ``Freund`` anmelden } & \mc{ Main Screen } \\
	  \mc{ Drawer Menu öffnen -> ``Offene Anfragen`` auswählen } & \mc{ Liste mit Anfragen erscheint } \\
	  \mc{ Anfrage ablehnen } & \mc{ Zurück auf Hauptbildschirm, Freund wird nicht angezeigt } \\
	  \hline
	\end{tabular}
\end{center}







\begin{center}
	\begin{tabular}{ | l | l | l | l | }
	\hline
	  \textbf{ID} & \textbf{Name} & \textbf{Priorität} & \textbf{Abdeckung} \\
	  \hline
	  TC04 & Freundschaftsanfrage (Annehmen) & Hoch & UC02 \\
	  \hline
	  \mc{\textbf{Vorbedingung}} & \mc{\textbf{Nachbedingung}} \\
	  \hline
	  \mc{Angemeldet, 2. Account bereit (Freund)} & \mc{Anfrage versendet und bestätigt} \\
	  \hline
	  \mc{\textbf{Schritte}} & \mc{\textbf{Erwartet}} \\
	  \hline
	  \mc{ Drawer Menu öffnen -> ``Freund hinzufügen`` auswählen } & \mc{Formular erscheint} \\
	  \mc{ E-Mail des Freundes eingeben, Einladen auswählen } & \mc{Einladung versendet, zurück auf Hauptbildschirm} \\
	  \mc{ Drawer Menu öffnen -> ``Abmelden`` auswählen } & \mc{ Register/Login erscheint } \\
	  \mc{ Mit ``Freund`` einloggen } & \mc{ Main Screen } \\
	  \mc{ Drawer Menu öffnen -> ``Offene Anfragen`` auswählen } & \mc{ Liste mit Anfragen erscheint } \\
	  \mc{ Anfrage ablehnen } & \mc{ Zurück auf Hauptbildschirm, Freund wird nicht angezeigt } \\
	  \hline
	\end{tabular}
\end{center}

\begin{center}
	\begin{tabular}{ | l | l | l | l | }
	\hline
	  \textbf{ID} & \textbf{Name} & \textbf{Priorität} & \textbf{Abdeckung} \\
	  \hline
	  TC05 & Freundschaft Auflösen & Hoch & UC03 \\
	  \hline
	  \mc{\textbf{Vorbedingung}} & \mc{\textbf{Nachbedingung}} \\
	  \hline
	  \mc{Angemeldet, mind 1 Freund} & \mc{Freundschaft aufgelöst} \\
	  \hline
	  \mc{\textbf{Schritte}} & \mc{\textbf{Erwartet}} \\
	  \hline
	  \mc{ Freund aus Liste auswählen } & \mc{Freund Details werden angezeigt} \\
	  \mc{ Reiter Details ``Freund entfernen`` auswählen } & \mc{Benachrichtigung, zurück auf Hauptbildschirm, Freund erscheint nicht mehr in der Liste} \\
	  \hline
	\end{tabular}
\end{center}

\begin{center}
	\begin{tabular}{ | l | l | l | l | }
	\hline
	  \textbf{ID} & \textbf{Name} & \textbf{Priorität} & \textbf{Abdeckung} \\
	  \hline
	  TC06 & Aktuelle Position Abfragen & Hoch & UC04 \\
	  \hline
	  \mc{\textbf{Vorbedingung}} & \mc{\textbf{Nachbedingung}} \\
	  \hline
	  \mc{Angemeldet, mind 1 Freund} & \mc{Position des Freundes bekannt} \\
	  \hline
	  \mc{\textbf{Schritte}} & \mc{\textbf{Erwartet}} \\
	  \hline
	  \mc{ Freund aus Liste auswählen } & \mc{Freund Details werden angezeigt} \\
	  \mc{ Reiter Details Distanz } & \mc{Aktuelle geschätze Distanz wird angezeigt} \\
	  \mc{ Reiter Karte auswählen } & \mc{Aktuelle Position wird auf der Karte angezeigt} \\
	  \hline
	\end{tabular}
\end{center}

\begin{center}
	\begin{tabular}{ | l | l | l | l | }
	\hline
	  \textbf{ID} & \textbf{Name} & \textbf{Priorität} & \textbf{Abdeckung} \\
	  \hline
	  TC07 & Proximety Alarm & Hoch & UC05, UC08 \\
	  \hline
	  \mc{\textbf{Vorbedingung}} & \mc{\textbf{Nachbedingung}} \\
	  \hline
	  \mc{Angemeldet, mind 1 Freund, Freund hat bekannte Position} & \mc{Proximety Alarm ausgelöst} \\
	  \hline
	  \mc{\textbf{Schritte}} & \mc{\textbf{Erwartet}} \\
	  \hline
	  \mc{ Android Mock Location far far away (muss mer no gugge wie da gnau gaht...) } & \mc{Kein Alarm} \\
	  \mc{ Distanz für Alarm 10 km einstellen } & \mc{} \\
	  \mc{ Android Mock Location close ~8 km } & \mc{Alarm} \\
	  \mc{ Android Mock Location far away } & \mc{Kein Alarm} \\
	  \mc{ Distanz für Alarm reduzieren auf 3 km } & \mc{} \\
	  \mc{ Android Mock Location very close } & \mc{Alarm} \\
	  \hline
	\end{tabular}
\end{center}

\begin{center}
	\begin{tabular}{ | l | l | l | l | }
	\hline
	  \textbf{ID} & \textbf{Name} & \textbf{Priorität} & \textbf{Abdeckung} \\
	  \hline
	  TC08 & Freund kontaktieren & Tief & UC06 \\
	  \hline
	  \mc{\textbf{Vorbedingung}} & \mc{\textbf{Nachbedingung}} \\
	  \hline
	  \mc{Angemeldet, mind 1 Freund} & \mc{E-Mail an Freund gesendet} \\
	  \hline
	  \mc{\textbf{Schritte}} & \mc{\textbf{Erwartet}} \\
	  \hline
	  \mc{ Freund aus Liste auswählen } & \mc{Freund Details werden angezeigt} \\
	  \mc{ Kontakt Icon auswählen } & \mc{Externe Applikation für E-Mail wird gestartet (Bsp. GMail), E-Mail von Freund eingetragen} \\
	  \mc{ Nachricht schreiben und absenden } & \mc{Freund erhält geschrieben Nachricht} \\
	  \hline
	\end{tabular}
\end{center}

\begin{center}
	\begin{tabular}{ | l | l | l | l | }
	\hline
	  \textbf{ID} & \textbf{Name} & \textbf{Priorität} & \textbf{Abdeckung} \\
	  \hline
	  TC09 & Freund-Spezifische Einstellungen & Mittel & UC07 \\
	  \hline
	  \mc{\textbf{Vorbedingung}} & \mc{\textbf{Nachbedingung}} \\
	  \hline
	  \mc{Angemeldet, mind 1 Freund} & \mc{Einstellungen angepasst} \\
	  \hline
	  \mc{\textbf{Schritte}} & \mc{\textbf{Erwartet}} \\
	  \hline
	  \mc{ Freund aus Liste auswählen } & \mc{Freund Details werden angezeigt} \\
	  \mc{ Alarm abschalten } & \mc{} \\
	  \mc{ TC07 ausführen } & \mc{Es wird kein Alarm ausgelöst} \\
	  \hline
	\end{tabular}
\end{center}

\begin{center}
	\begin{tabular}{ | l | l | l | l | }
	\hline
	  \textbf{ID} & \textbf{Name} & \textbf{Priorität} & \textbf{Abdeckung} \\
	  \hline
	  TC10 & Allgemeine Einstellungen & Mittel & UC08 \\
	  \hline
	  \mc{\textbf{Vorbedingung}} & \mc{\textbf{Nachbedingung}} \\
	  \hline
	  \mc{Angemeldet} & \mc{Einstellungen angepasst} \\
	  \hline
	  \mc{\textbf{Schritte}} & \mc{\textbf{Erwartet}} \\
	  \hline
	  \mc{ Drawer Menu öffnen -> ``Einstellungen`` auswählen } & \mc{Einstellungen erscheinen} \\
	  \mc{ Distanz auf 11 km ändern } & \mc{} \\
	  \mc{ Allg-Alarm ausschalten } & \mc{} \\
	  \mc{ Abmelden, App schliessen } & \mc{} \\
	  \mc{ App starten, Anmelden } & \mc{Hauptbildschirm erscheint} \\
	  \mc{ Drawer Menu öffnen -> ``Einstellungen`` auswählen } & \mc{Einstellungen gespeichert, wie in Schritt 2,3} \\
	  \hline
	\end{tabular}
\end{center}

\subsection{Test Resulate}
Hier werden die Resulatate der Ausführungen mit der verwendeten Umgebung aufgeführt werden.

\begin{center}
	\begin{tabular}{ | l | l | l | l | l | }
	\hline
	Name / Id & Datum & Umgebung & Resultat (OK/NOK) & Bemerkungen \\
	\hline
	 & & & & \\
	\hline
	\end{tabular}
\end{center}



