% Definieren von Tabellen und Boxes Umgebung
% UseCase Tables
\newenvironment{usecase}%
{%
\setlength{\tabcolsep}{5pt}
\setlist[itemize]{noitemsep,topsep=0pt,partopsep=0pt}
\setlist[enumerate]{noitemsep,topsep=0pt,partopsep=0pt}
%\renewcommand{\arraystretch}{2.5}
\centering
\begin{longtable}[H]{|l|p{9cm}|}
\arrayrulecolor{black}
\hline
}{%
\end{longtable}
}%
\newcommand{\ucElement}[2]{%
\textbf{#1} & \mbox{}\par\vspace{-\baselineskip}#2\\
\hline
}
% Tabellen definieren
\newenvironment{deftab}[3]%
{%
\setlength{\tabcolsep}{20pt}
\renewcommand{\arraystretch}{1.5}
\def\cap{#2}
\def\lab{#3}
\begin{table}[!htbp]
\footnotesize
\begin{center}
\begin{longtable}{#1}
}{%
\end{longtable}
\end{center}
\caption{\cap} % Text unter Tabelle
\label{table:\lab} % Referenz der Tabelle \ref{table:xyz}
\end{table}
\FloatBarrier % Sicherstellen das alles weitere nach der Tabelle kommt
}%
% -------------------------------------------------------
% Boxes
% Warnbox
\newsavebox{\wbox}
\newenvironment{warnbox}[1][]
  {%
   \setlength{\fboxsep}{8pt}
   \setlength{\fboxrule}{1pt}
   \footnotesize
   \newcommand\colboxcolor{#1}%
   \begin{lrbox}{\wbox}%
   \begin{minipage}[c]{0.9\textwidth}%   
 }{%
  \end{minipage}
  \end{lrbox}%
   \begin{center}
   \fcolorbox{warning-line}{warning}{\usebox{\wbox}}
\end{center}}
% Errorbox
\newsavebox{\ebox}
\newenvironment{errorbox}[1][]
  {%
   \setlength{\fboxsep}{8pt}
   \setlength{\fboxrule}{1pt}
   \footnotesize
   \newcommand\colboxcolor{#1}%
   \begin{lrbox}{\ebox}%
   \begin{minipage}[c]{0.9\textwidth}%
 }{%
  \end{minipage}
  \end{lrbox}%
   \begin{center}
   \fcolorbox{error-line}{error}{\usebox{\ebox}}
\end{center}}
%-- Tabellen ----------------------------------------------
\newcolumntype{+}{>{\global\let\currentrowstyle\relax}}
\newcolumntype{^}{>{\currentrowstyle}}
\newcommand{\rowstyle}[1]{\gdef\currentrowstyle{#1}%
#1\ignorespaces
}
\newcolumntype{L}[1]{>{\raggedright\arraybackslash}p{#1}} % linksbündig mit Breitenangabe
\newcolumntype{C}[1]{>{\centering\arraybackslash}p{#1}} % zentriert mit Breitenangabe
\newcolumntype{R}[1]{>{\raggedleft\arraybackslash}p{#1}} % rechtsbündig mit Breitenangabe
\newcommand{\ltab}{\raggedright\arraybackslash} % Tabellenabschnitt linksbündig
\newcommand{\ctab}{\centering\arraybackslash} % Tabellenabschnitt zentriert
\newcommand{\rtab}{\raggedleft\arraybackslash} % Tabellenabschnitt rechtsbündig