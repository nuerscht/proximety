%----------------------------------------------------------
% Allgemeine Latex commands, enviroments etc...
% gehören in dieses Dokumentenfragment.
%----------------------------------------------------------
%-- Renew paragraph command -------------------------------
\makeatletter
\renewcommand\paragraph{%
   \@startsection{paragraph}{4}{0mm}%
      {-\baselineskip}%
      {.5\baselineskip}%
      {\normalfont\normalsize\bfseries}}
\makeatother
%-- Abbildungsverzeichnis ---------------------------------
\renewcommand{\listoffigures}{%
\begingroup
  \tocsection
  \tocfile{\listfigurename}{lof}
\endgroup}
%-- Abbildungsverzeichnis ---------------------------------
\renewcommand{\listoftables}{%
\begingroup
  \tocsection
  \tocfile{\listtablename}{lot}
\endgroup}
%-- Listings ---------------------------------------------
\renewcommand{\lstlistingname}{Code}
\renewcommand{\lstlistlistingname}{Codelistings}
\renewcommand{\lstlistoflistings}{%
\begingroup
  \tocsection
  \tocfile{\lstlistlistingname}{lol}
\endgroup}

%-- Aufgabe mit Trennlinie --------------------------------
\newcounter{cntaufgabe}
\newcounter{cntfrage}
\setcounter{cntfrage}{1}
\newcommand{\resetaufgabe}{%
\par
\vspace{10mm}
\leftskip=0cm
\setcounter{cntfrage}{1}
}
\newcommand{\aufgabe}[1]{%
\stepcounter{cntaufgabe}
\resetaufgabe
\vspace{2ex}
\begin{tabular}{ r p{0.8\textwidth} }
\textbf{\thecntaufgabe} &
\textbf{#1}
\end{tabular}
\vspace{1mm}
\hrule
\vspace{2ex}%
}
%-- Frage mit Trennlinie --------------------------------
\newcommand{\frage}[1]{%
\par
 \vspace{5mm}
\leftskip=1.4cm
\begin{tabular}{ r p{0.8\textwidth} }
\alph{cntfrage}) &
\textit{#1}
\end{tabular}
\vspace{1mm}
\par
\leftskip=2.35cm
\stepcounter{cntfrage}
}
%-- new paragraph ------------
\newcommand{\vpar}{
\par
\vspace{1ex}
}
%-- quick multicolumn ------
\newcommand{\mc}[1]{
	\multicolumn{2}{|p{0.49\textwidth}|}{#1}
}
