% Anpassung an deutsche Sprache
% Umlaute ----------------------------------------------------------------------
%   Umlaute/Sonderzeichen wie äüöß direkt im Quelltext verwenden (CodePage).
%   Erlaubt automatische Trennung von Worten mit Umlauten.
% ------------------------------------------------------------------------------
\usepackage[utf8]{inputenc}
\usepackage[T1]{fontenc}
\usepackage[ngerman]{babel}

% Code support
\usepackage{listings}

% Seitenränder und Zeilenabstände
\usepackage{geometry}
\usepackage{setspace}

% Schriften und Farben
\usepackage{color}
\usepackage{xcolor}
\usepackage{colortbl} % Für farbige Tabellen
\usepackage[default,osfigures,scale=1]{opensans} % Select font

% Textumgebungen
\usepackage{multicol} % Text in mehreren Kolumnen darstellen

% Tabellen
\usepackage{hhline}
\usepackage{array}
\usepackage{longtable}

% Floatbarrier 
\usepackage{placeins} 
\usepackage{float}
\restylefloat{table}
% Hyperrefs und PDF Information
\usepackage{hyperref}
\usepackage{caption}
\usepackage{subcaption}

% Tabellen
\usepackage{booktabs}

% Inhaltsverzeichnis umbenennen
\usepackage{tocbibind}

% Listen
\usepackage[ampersand]{easylist} % Einfache Liste Umgebung
% http://en.wikibooks.org/wiki/LaTeX/List_Structures
\usepackage{enumitem}

% Kopf und Fusszeilen
\usepackage{fancyhdr}

% Grafiken und Textbausteine
\usepackage[absolute]{textpos} % Absolue Positionierung
%\usepackage[dvips,final]{graphicx} % Einbinden von JPG-Grafiken ermöglichen für
%dvi format
\usepackage[pdftex]{graphicx}
\usepackage{graphics} % keepaspectratio
\usepackage{floatflt} % zum Umfließen von Bildern
\graphicspath{{bilder/}} % hier liegen die Bilder des Dokuments

% Mathe
\usepackage{amsmath}
\usepackage{amssymb}

% Acronym Support
\usepackage[printonlyused]{acronym}