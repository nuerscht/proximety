Wir haben uns viel vorgenommen - das war uns schon zu Beginn der Arbeit bewusst. Erfahrung in der Programmierung war bei allen Teammitgliedern vorhanden; Erfahrung in der Android-Programmierung fehlte allerdings fast komplett.

Im Moodle-Kurs wurden einige App-Vorschläge gemacht, welchen bei uns nicht auf fruchtbaren Boden fielen. So wurde überlegt, welches Thema unsere App behandeln könnte und schnell war man sich einig ein Überwachungsprogramm zu entwickeln. So konnte durch die Vielfalt der eingesetzten Module während der Programmierung eine Menge über Android gelernt werden und als netter Nebeneffekt müssen wir uns in Zukunft nicht mehr Fragen, ob unsere Kommilitonen überhaupt schon auf dem Weg in die FFHS sind.

Die ersten vom Dozenten gestellten Aufgaben waren schnell gelöst, wurden wir doch durch selbigen gut in diesen Themen geschult. Ein Zeitmanagement war schon dadurch gegeben, dass im vier Wochen Rhythmus Zwischenstände abgegeben werden mussten. 

Die Programmierung nahm in unserem Fall einiges mehr Zeit in Anspruch als das Schreiben der Dokumentation, da wir sowohl die App als auch den Server-Teil zu implementieren hatten. Zu den bereits in der Analyse dokumentierten Features haben wir noch ein paar ergänzt. So ist es nun zum Beispiel möglich mit dem aus Android bekannten Swipe eine Activity neu zu laden. 

Als wir die Google Map in Betrieb nehmen wollten, sind wir über Eigenheiten der Eclipse Entwicklungsumgebung mit Android gestossen. So musste ein zusätzliches Projekt ins Eclipse importiert werden, damit die Google Services gebraucht werden konnten. Diesen Umstand fanden wir etwas irritierend und da zur gleichen Zeit das Android Studio freigeben wurde, haben wir uns entschieden Eclipse zu verlassen und auf Android Studio mit Gradle Support umzuschwenken. Was für eine Wohltat!

Natürlich durften wir auch mit dem einen oder anderen Problem kämpfen - Layout (nach hartem Kampf sind wir den Mock-Ups doch sehr nah), ActionBar (deprecated) oder Google Cloud Messaging (initales Setup) um nur einige aufzuzählen.

Doch nun sind wir in den letzten Zügen. Die App ist fertig! Die Präsentation erstellt! Wir freuen uns darauf unsere Resultate zu präsentieren!